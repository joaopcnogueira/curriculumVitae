%% start of file `template.tex'.
%% Copyright 2006-2013 Xavier Danaux (xdanaux@gmail.com).
%
% This work may be distributed and/or modified under the
% conditions of the LaTeX Project Public License version 1.3c,
% available at http://www.latex-project.org/lppl/.


\documentclass[12pt,a4paper,roman]{moderncv}        % possible options include font size ('10pt', '11pt' and '12pt'), paper size ('a4paper', 'letterpaper', 'a5paper', 'legalpaper', 'executivepaper' and 'landscape') and font family ('sans' and 'roman')

% modern themes
\moderncvstyle{banking}                            % style options are 'casual' (default), 'classic', 'oldstyle' and 'banking'
\moderncvcolor{black}                                % color options 'blue' (default), 'orange', 'green', 'red', 'purple', 'grey' and 'black'
%\renewcommand{\familydefault}{\sfdefault}         % to set the default font; use '\sfdefault' for the default sans serif font, '\rmdefault' for the default roman one, or any tex font name
%\nopagenumbers{}                                  % uncomment to suppress automatic page numbering for CVs longer than one page

% character encoding
\usepackage[utf8]{inputenc}                       % if you are not using xelatex ou lualatex, replace by the encoding you are using
%\usepackage{CJKutf8}                              % if you need to use CJK to typeset your resume in Chinese, Japanese or Korean

% adjust the page margins
\usepackage[scale=0.81]{geometry}
%\setlength{\hintscolumnwidth}{3cm}                % if you want to change the width of the column with the dates
%\setlength{\makecvtitlenamewidth}{10cm}           % for the 'classic' style, if you want to force the width allocated to your name and avoid line breaks. be careful though, the length is normally calculated to avoid any overlap with your personal info; use this at your own typographical risks...

\usepackage{import}

% personal data
\name{João Paulo da Costa}{Nogueira}
%\title{Curriculum Vitae}                               % optional, remove / comment the line if not wanted
% \address{my address, line 1, line 2, line 3, postcode}{}{}% optional, remove / comment the line if not wanted; the "postcode city" and and "country" arguments can be omitted or provided empty
\phone[mobile]{+55 85 9 9604 5233}                   % optional, remove / comment the line if not wanted
% \phone[fixed]{01234 123456}                    % optional, remove / comment the line if not wanted
%\phone[fax]{+3~(456)~789~012}                      % optional, remove / comment the line if not wanted
\email{joaonogueira@fisica.ufc.br}                               % optional, remove / comment the line if not wanted
\social[github]{joaopcnogueira}
\social[linkedin]{joaopaulonogueira}
%\homepage{}                         % optional, remove / comment the line if not wanted

\usepackage{xcolor}
\definecolor{gray-black}{HTML}{62676a}

%\extrainfo{additional information}                 % optional, remove / comment the line if not wanted
%\photo[64pt][0.4pt]{picture}                       % optional, remove / comment the line if not wanted; '64pt' is the height the picture must be resized to, 0.4pt is the thickness of the frame around it (put it to 0pt for no frame) and 'picture' is the name of the picture file
%\quote{Some quote}                                 % optional, remove / comment the line if not wanted

% to show numerical labels in the bibliography (default is to show no labels); only useful if you make citations in your resume
%\makeatletter
%\renewcommand*{\bibliographyitemlabel}{\@biblabel{\arabic{enumiv}}}
%\makeatother
%\renewcommand*{\bibliographyitemlabel}{[\arabic{enumiv}]}% CONSIDER REPLACING THE ABOVE BY THIS

% bibliography with mutiple entries
%\usepackage{multibib}
%\newcites{book,misc}{{Books},{Others}}
%----------------------------------------------------------------------------------
%            content
%----------------------------------------------------------------------------------
\begin{document}
%\begin{CJK*}{UTF8}{gbsn}                          % to typeset your resume in Chinese using CJK
%-----       resume       ---------------------------------------------------------
\makecvtitle

\small{I want to purse a Data Science career. I am a PhD student on Complex Systems, working with spread of epidemics by data analysis and computer modeling. I am proficient at Machine Learning and have been exposed to Neural Networks and Deep Learning. I have also studied Econophysics, being exposed to Risk Management, Markowitz Efficient Frontier and Black-Scholes model.}

\section{EDUCATION}

\vspace{6pt}

\cventry{\textcolor{black}{\textcolor{gray-black}{since Ago 2016 - Actual}}}{Universidade Federal do  Ceará}{PhD in Physics, Complex Systems}{Ceará, BR}{}{}

\vspace{6pt}

\cventry{\textcolor{gray-black}{Ago 2014 – Jul 2016}}{Universidade Federal do  Ceará}{Master in Physics, Complex Systems}{Ceará, BR}{}{}\textcolor{black}{Master thesis: Rugosidade em Bilhares Clássicos.\\(\textit{Dynamical Systems and Chaos}).}

\vspace{6pt}

\cventry{\textcolor{gray-black}{Mar 2012 – Feb 2016}}{Universidade Federal do  Ceará}{Bachelor in Physics}{Ceará, BR}{}{}\textcolor{black}{Bachelor thesis: Análise Qualitativa da Integrabilidade de Bilhares.\\(\textit{Dynamical Systems and Chaos}). }

\section{EMPLOYMENT}
My daily activities involve data analysis, computational modeling in complex networks, development of algorithms and scripts to automate simulation processes, result reporting, scientific writing, seminars presentation and physics lectures to freshmen.
\section{SKILLS}

\textbf{Data Analysis:} Python, Pandas, NumPy, NetworkX, MySQL, Excel, R.

\vspace{6pt}

\textbf{Data Visualization:} Matplotlib, Seaborn, Plotly, ggplot2.

\vspace{6pt}

\textbf{Machine Learning:} Scikit-Learn.

\vspace{6pt}

\textbf{Version Control:} Git, Github.

\vspace{6pt}

\textbf{Languages:} Portuguese (native), English (fluent).

%\section{PROJECTS}
%
%\vspace{6pt}
%
%\begin{itemize}
%
%\item{I was a "fresher representative" in my 2nd and 3rd years of university, this required me to guide, look after, and ensure that a particular flat of first years have a good time in their first week, and feel consoled in what for most of them is there first time living away from home. We were responsible for the safety and wellbeing of the group of first years during the first week, and during this time I made good friends with all of them.}
%
%\vspace{6pt}
%
%\item{I am a member of a number of university societies. I was also the vice president and co-founder of the flash mob society. My roles in this included recruiting members, in which during "fresher's fair" we enlisted over 200 new members. This was regarded as very successful, considering other societies averaged around 50. I also appeared in an interview on the university television station, set up a society bank account, and helped organise the events. One of these events was featured in the local newspaper.}
%
%\vspace{6pt}
%
%\item{I am also an avid hiker, having completed the national 3 peaks challenge last summer. Other interest include guitar, which I am self-taught, and home brewing.}
%
%\end{itemize}

%\section{CONTACT ME}
%
%\vspace{6pt}
% 
%\begin{itemize}
%
%\item {emanuelfontelles@hotmail.com}                               % optional, remove / comment the line if not wanted
%
%\vspace{6pt}
%
%\item{emanuelfontelles.github.io/}
%
%\end{itemize}

% Publications from a BibTeX file without multibib
%  for numerical labels: \renewcommand{\bibliographyitemlabel}{\@biblabel{\arabic{enumiv}}}% CONSIDER MERGING WITH PREAMBLE PART
%  to redefine the heading string ("Publications"): \renewcommand{\refname}{Articles}
\nocite{*}
\bibliographystyle{plain}
\bibliography{publications}                        % 'publications' is the name of a BibTeX file

% Publications from a BibTeX file using the multibib package
%\section{Publications}
%\nocitebook{book1,book2}
%\bibliographystylebook{plain}
%\bibliographybook{publications}                   % 'publications' is the name of a BibTeX file
%\nocitemisc{misc1,misc2,misc3}
%\bibliographystylemisc{plain}
%\bibliographymisc{publications}                   % 'publications' is the name of a BibTeX file

%-----       letter       ---------------------------------------------------------

\end{document}


%% end of file `template.tex'.
