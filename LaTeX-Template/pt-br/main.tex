\documentclass[12pt,a4paper,roman]{moderncv}        % possible options include font size ('10pt', '11pt' and '12pt'), paper size ('a4paper', 'letterpaper', 'a5paper', 'legalpaper', 'executivepaper' and 'landscape') and font family ('sans' and 'roman')

% modern themes
\moderncvstyle{banking}                            % style options are 'casual' (default), 'classic', 'oldstyle' and 'banking'
\moderncvcolor{blue}                                % color options 'blue' (default), 'orange', 'green', 'red', 'purple', 'grey' and 'black'

\usepackage{xcolor}
\definecolor{gray-black}{HTML}{62676a}

%\renewcommand{\familydefault}{\sfdefault}         % to set the default font; use '\sfdefault' for the default sans serif font, '\rmdefault' for the default roman one, or any tex font name
%\nopagenumbers{}                                  % uncomment to suppress automatic page numbering for CVs longer than one page

% character encoding
\usepackage[utf8]{inputenc}                       % if you are not using xelatex ou lualatex, replace by the encoding you are using

% adjust the page margins
\usepackage[scale=0.81]{geometry}
%\setlength{\hintscolumnwidth}{3cm}                % if you want to change the width of the column with the dates
%\setlength{\makecvtitlenamewidth}{10cm}           % for the 'classic' style, if you want to force the width allocated to your name and avoid line breaks. be careful though, the length is normally calculated to avoid any overlap with your personal info; use this at your own typographical risks...

\usepackage{import}

% personal data
\name{Emanuel}{Pinheiro Fontelles}
%\title{Curriculum Vitae}                               % optional, remove / comment the line if not wanted
% \address{my address, line 1, line 2, line 3, postcode}{}{}% optional, remove / comment the line if not wanted; the "postcode city" and and "country" arguments can be omitted or provided empty
\phone[mobile]{+55 85 9 8958 0740}                   % optional, remove / comment the line if not wanted
% \phone[fixed]{01234 123456}                    % optional, remove / comment the line if not wanted
%\phone[fax]{+3~(456)~789~012}                      % optional, remove / comment the line if not wanted
\email{emanuelfontelles@hotmail.com}                               % optional, remove / comment the line if not wanted
\social[github]{emanuelfontelles}
\social[linkedin]{emanuelfontelles}
\homepage{emanuelfontelles.github.io/}                         % optional, remove / comment the line if not wanted
%\extrainfo{additional information}                 % optional, remove / comment the line if not wanted
%\photo[64pt][0.4pt]{picture}                       % optional, remove / comment the line if not wanted; '64pt' is the height the picture must be resized to, 0.4pt is the thickness of the frame around it (put it to 0pt for no frame) and 'picture' is the name of the picture file
%\quote{Some quote}                                 % optional, remove / comment the line if not wanted

% to show numerical labels in the bibliography (default is to show no labels); only useful if you make citations in your resume
%\makeatletter
%\renewcommand*{\bibliographyitemlabel}{\@biblabel{\arabic{enumiv}}}
%\makeatother
%\renewcommand*{\bibliographyitemlabel}{[\arabic{enumiv}]}% CONSIDER REPLACING THE ABOVE BY THIS

% bibliography with mutiple entries
%\usepackage{multibib}
%\newcites{book,misc}{{Books},{Others}}
%----------------------------------------------------------------------------------
%            content
%----------------------------------------------------------------------------------

\begin{document}
%-----       resume       ---------------------------------------------------------
\makecvtitle

\small{Como aluno de doutorado em Física, venho trabalhando com Redes Complexas, Econofísica e Mecânica Estatística, modelando diversos sistemas físicos. Minha formação é em Física e Mecânica Estatística. Tenho habilidades em linguagens de programação, algoritmos de aprendizagem de máquina, mineração de dados e visualização. Estou procurando por uma posição de cientista de dados para explorar minhas habilidades como cientista como também analista de dados.}

\section{FORMAÇÃO}

\vspace{6pt}

\cventry{\textcolor{black}{\textcolor{gray-black}{desde Março de 2018}}}{Universidade Federal do  Ceará}{Doutorado em Física, Sistemas Complexos}{Ceará, BR}{}{}

\vspace{6pt}

\cventry{\textcolor{gray-black}{Março 2016 – Fevereiro 2018}}{Universidade Federal do  Ceará}{Mestrado em Física, Sistemas Complexos}{Ceará, BR}{}{}\textcolor{black}{Título: Estudo de redes complexas e Mecânica Estatística não-extensiva.}  % arguments 3 to 6 can be left empty

\vspace{6pt}

\cventry{\textcolor{gray-black}{Março 2012 – Fevereiro 2016}}{Universidade Federal do  Ceará}{Bacharelado em Física}{Ceará, BR}{}{}{
\textcolor{black}{Título: Percolação e Criticalidade Auto-Organizada.}}% arguments 3 to 6 can be left empty

% \end{itemize}
\section{EXPERIÊNCIA}

\vspace{6pt}
Diariamente desenvolvo projetos em redes complexas, desde a criação de algoritmos, análise de dados, reporte de resultados e escrita científica. Nos projetos realizados desenvolvi modelos que atuam entre Mecânica Estatistíca e Redes Complexas.
\vspace{6pt}

\cventry{\textcolor{gray-black}{desde Março de 2018}}{Universidade Federal do Ceará}{Doutorado em Física}{Ceará, BR}{}{}

\section{HABILIDADES}

\vspace{6pt}
Atuo dentro do ecossistema de Python, trabalhando com as principais bibliotecas na área de computação científica e tratamento de dados. As mesma compreendem algébra linear, exploração de dados, visualização e aprendizado de máquina. Também possuo familiaridade com diversos softwares e ferramentas de desenvolvimento \textit{web}.
\vspace{6pt}

\textbf{Análise de Dados:} Pandas, Scikit-Learn, NumPy, NetworkX, SciPy, Jupyter e exposição a Tensorflow and Keras.

\vspace{6pt}

 \textbf{Visualização de Dados:} Matplotlib, Seaborn, Plotly, Bokeh, Altair e ligeira exposição a D3.js.

\vspace{6pt}

\textbf{Outras ferramentas:} C/C++, R, Julia, Git, GitHub, Markdown, Jekyll, Ruby on Rails, YAML, JSON, SQL, CSV, XML, HTML/CSS,  Javascript, Linux, Bash, SSH e Windows.

\vspace{6pt}

\textbf{Idiomas:} Português (nativo), Inglês (fluente) e noções de Espanhol.

%\section{PROJECTS}
%
%\vspace{6pt}
%
%\begin{itemize}
%
%\item{I was a "fresher representative" in my 2nd and 3rd years of university, this required me to guide, look after, and ensure that a particular flat of first years have a good time in their first week, and feel consoled in what for most of them is there first time living away from home. We were responsible for the safety and wellbeing of the group of first years during the first week, and during this time I made good friends with all of them.}
%
%\vspace{6pt}
%
%\item{I am a member of a number of university societies. I was also the vice president and co-founder of the flash mob society. My roles in this included recruiting members, in which during "fresher's fair" we enlisted over 200 new members. This was regarded as very successful, considering other societies averaged around 50. I also appeared in an interview on the university television station, set up a society bank account, and helped organise the events. One of these events was featured in the local newspaper.}
%
%\vspace{6pt}
%
%\item{I am also an avid hiker, having completed the national 3 peaks challenge last summer. Other interest include guitar, which I am self-taught, and home brewing.}
%
%\end{itemize}

%\section{CONTACT ME}
%
%\vspace{6pt}
% 
%\begin{itemize}
%
%\item {emanuelfontelles@hotmail.com}                               % optional, remove / comment the line if not wanted
%
%\vspace{6pt}
%
%\item{emanuelfontelles.github.io/}
%
%\end{itemize}

% Publications from a BibTeX file without multibib
%  for numerical labels: \renewcommand{\bibliographyitemlabel}{\@biblabel{\arabic{enumiv}}}% CONSIDER MERGING WITH PREAMBLE PART
%  to redefine the heading string ("Publications"): \renewcommand{\refname}{Articles}
\nocite{*}
\bibliographystyle{plain}
\bibliography{publications}                        % 'publications' is the name of a BibTeX file

% Publications from a BibTeX file using the multibib package
%\section{Publications}
%\nocitebook{book1,book2}
%\bibliographystylebook{plain}
%\bibliographybook{publications}                   % 'publications' is the name of a BibTeX file
%\nocitemisc{misc1,misc2,misc3}
%\bibliographystylemisc{plain}
%\bibliographymisc{publications}                   % 'publications' is the name of a BibTeX file

%-----       letter       ---------------------------------------------------------

\end{document}


%% end of file `template.tex'.
